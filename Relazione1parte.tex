\documentclass[12pt]{article}
\title{Risolutore di puzzle – Parte 1}
\author{Alberto Andeliero}
\date{21 gennaio 2015}
\begin{document}
\maketitle
\centerline{Programmazione concorrente e distribuita}
\centerline{Progetto A.A. 2014/2015}
\section{Scelte progettuali}
PuzzleSolver è stato concepito per risolvere puzzle rettangolari, estendibile con altri puzzle rettangolari che contengano altri oggetti anzichè Stringhe al proprio interno. Quindi oltre alla classe PuzzleSolver che conterrà il main dell'applicazione, è stata modellata la classe Puzzle che è responsabile delle operazioni quali lettura da file di input, scrittura in file di output, e fondamentale il metodo sort che riordina i tasselli del puzzle secondo quanto richiesto dalla specifica tecnica. La gerarchia della classe puzzle non è 
costituita solamente da essa perchè non è previsto avere puzzle di altre forme. 
\section{Organizzazione delle classi}
\section{Algoritmo di risoluzione}
\section{Corettezza del programma}

\end{document}